% Options for packages loaded elsewhere
\PassOptionsToPackage{unicode}{hyperref}
\PassOptionsToPackage{hyphens}{url}
%
\documentclass[
]{article}
\usepackage{lmodern}
\usepackage{amssymb,amsmath}
\usepackage{ifxetex,ifluatex}
\ifnum 0\ifxetex 1\fi\ifluatex 1\fi=0 % if pdftex
  \usepackage[T1]{fontenc}
  \usepackage[utf8]{inputenc}
  \usepackage{textcomp} % provide euro and other symbols
\else % if luatex or xetex
  \usepackage{unicode-math}
  \defaultfontfeatures{Scale=MatchLowercase}
  \defaultfontfeatures[\rmfamily]{Ligatures=TeX,Scale=1}
\fi
% Use upquote if available, for straight quotes in verbatim environments
\IfFileExists{upquote.sty}{\usepackage{upquote}}{}
\IfFileExists{microtype.sty}{% use microtype if available
  \usepackage[]{microtype}
  \UseMicrotypeSet[protrusion]{basicmath} % disable protrusion for tt fonts
}{}
\makeatletter
\@ifundefined{KOMAClassName}{% if non-KOMA class
  \IfFileExists{parskip.sty}{%
    \usepackage{parskip}
  }{% else
    \setlength{\parindent}{0pt}
    \setlength{\parskip}{6pt plus 2pt minus 1pt}}
}{% if KOMA class
  \KOMAoptions{parskip=half}}
\makeatother
\usepackage{xcolor}
\IfFileExists{xurl.sty}{\usepackage{xurl}}{} % add URL line breaks if available
\IfFileExists{bookmark.sty}{\usepackage{bookmark}}{\usepackage{hyperref}}
\hypersetup{
  hidelinks,
  pdfcreator={LaTeX via pandoc}}
\urlstyle{same} % disable monospaced font for URLs
\usepackage{graphicx}
\makeatletter
\def\maxwidth{\ifdim\Gin@nat@width>\linewidth\linewidth\else\Gin@nat@width\fi}
\def\maxheight{\ifdim\Gin@nat@height>\textheight\textheight\else\Gin@nat@height\fi}
\makeatother
% Scale images if necessary, so that they will not overflow the page
% margins by default, and it is still possible to overwrite the defaults
% using explicit options in \includegraphics[width, height, ...]{}
\setkeys{Gin}{width=\maxwidth,height=\maxheight,keepaspectratio}
% Set default figure placement to htbp
\makeatletter
\def\fps@figure{htbp}
\makeatother
\setlength{\emergencystretch}{3em} % prevent overfull lines
\providecommand{\tightlist}{%
  \setlength{\itemsep}{0pt}\setlength{\parskip}{0pt}}
\setcounter{secnumdepth}{-\maxdimen} % remove section numbering

\author{}
\date{}

\begin{document}

Problem set for Ch.3

\begin{quote}
Kensuke Sawada

1H210301-3
\end{quote}

1. Using the data in GPA2 on 4,137 college students, the following
equation was estimated by OLS:

\emph{\^{}colgpa}= 1.392 - .0135 \emph{hsperc} + .00148 \emph{sat}

\emph{n}=4,137, \emph{R}\^{}2 = .273,

where colgpa is measured on a four-point scale, hsperc is the percentile
in the high school graduating class (defined so that, for example,
\emph{hsperc}= 5 means the top 5\% of the class), and sat is the
combined math and verbal scores on the student achievement test.

(i) Why does it make sense for the coefficient on \emph{hsperc} to be
negative?

Since the problem statement indicates that hsperc=5 means the top 5\% of
the class, the larger \emph{hsperc} represents lower grades in high
school. Hence, a negative coefficient for hsperc when other coefficients
are fixed means that low grades in high school make \emph{colgpa} lower.

(ii) What is the predicted college GPA when hsperc = 20 and = sat =
1,050?

\emph{colgpa}=1.392-.0135(20) +.00148(1050) =2.676

Substituting \emph{hsperc}=20 and \emph{sat}=1050 into the equation, it
obtains a college GPA of 2.676.

(iii) Suppose that two high school graduates, A and B, graduated in the
same percentile from high school, but Student A's SAT score was 140
points higher (about one standard deviation in the sample). What is the
predicted difference in college GPA for these two students? Is the
difference large?

Holding hsperc fixed, the difference in GPA is 140 times the coefficient
of sat.

.00148(140) =.2072

Hence, A is predicted to have a GPA .2072 higher than B.

(iv) Holding hsperc fixed, what difference in SAT scores leads to a
predicted colgpa difference of .50, or one-half of a grade point?
Comment on your answer.

Problem description show Δ\emph{colgpa} =.50 and
Δ\emph{colgpa}=.00148Δ\emph{sat}

.50=.00148Δ\emph{sat}

Δ\emph{sat}≈337.83

Hence, the difference of 337.83 SAT scores leads to a predicted colgpa
difference of .50. To predict 0.5 GPA difference, which is 12.5\% of the
total, would require a difference in SAT scores as large as 337.83.

C3.The file CEOSAL2 contains data on 177 chief executive officers and
can be used to examine the effects of firm performance on CEO salary.

(i)Estimate a model relating annual salary to firm sales and market
value. Make the model of the constant elasticity variety for both
independent variables. Write the results out in equation form.

\#Open the dataset

library(wooldridge)

data("ceosal2")

\#Model: log-log model

C3 \textless- lm(log(salary)\textasciitilde log(sales)+log(mktval),
data)

\#Report the result

library (stargazer)

stargazer (C3, type = "text", title = "Table 1: sales \&mktval")

Table 1: sales \&mktval

===============================================

Dependent variable:

-\/-\/-\/-\/-\/-\/-\/-\/-\/-\/-\/-\/-\/-\/-\/-\/-\/-\/-\/-\/-\/-\/-\/-\/-\/-\/-

log(salary)

-\/-\/-\/-\/-\/-\/-\/-\/-\/-\/-\/-\/-\/-\/-\/-\/-\/-\/-\/-\/-\/-\/-\/-\/-\/-\/-\/-\/-\/-\/-\/-\/-\/-\/-\/-\/-\/-\/-\/-\/-\/-\/-\/-\/-\/-\/-

log(sales) 0.162***

(0.040)

log(mktval) 0.107**

(0.050)

Constant 4.621***

(0.254)

-\/-\/-\/-\/-\/-\/-\/-\/-\/-\/-\/-\/-\/-\/-\/-\/-\/-\/-\/-\/-\/-\/-\/-\/-\/-\/-\/-\/-\/-\/-\/-\/-\/-\/-\/-\/-\/-\/-\/-\/-\/-\/-\/-\/-\/-\/-

Observations 177

R2 0.299

Adjusted R2 0.291

Residual Std. Error 0.510 (df = 174)

F Statistic 37.129*** (df = 2; 174)

===============================================

Note: *p\textless0.1; **p\textless0.05; ***p\textless0.01

Estimate log-log model to make the model of the constant elasticity
variety.

From Table 1, the estimated model is

log(\emph{salary})=.162 log(\emph{sales})+.107 log(\emph{mktval})+4.621

(ii)Add profits to the model from part (i). Why can this variable not be
included in logarithmic form? Would you say that these firm performance
variables explain most of the variation in CEO salaries?

\#model: add profits to C3

C3.1 \textless-
lm(log(salary)\textasciitilde log(sales)+log(mktval)+profits, data)

\#report the result

stargazer(C3.1, type = "text", title = "Table 2: sales\& mktval\&
profits")

Table 2: sales\& mktval\& profits

===============================================

Dependent variable:

-\/-\/-\/-\/-\/-\/-\/-\/-\/-\/-\/-\/-\/-\/-\/-\/-\/-\/-\/-\/-\/-\/-\/-\/-\/-\/-

log(salary)

-\/-\/-\/-\/-\/-\/-\/-\/-\/-\/-\/-\/-\/-\/-\/-\/-\/-\/-\/-\/-\/-\/-\/-\/-\/-\/-\/-\/-\/-\/-\/-\/-\/-\/-\/-\/-\/-\/-\/-\/-\/-\/-\/-\/-\/-\/-

log(sales) 0.161***

(0.040)

log(mktval) 0.098

(0.064)

profits 0.00004

(0.0002)

Constant 4.687***

(0.380)

-\/-\/-\/-\/-\/-\/-\/-\/-\/-\/-\/-\/-\/-\/-\/-\/-\/-\/-\/-\/-\/-\/-\/-\/-\/-\/-\/-\/-\/-\/-\/-\/-\/-\/-\/-\/-\/-\/-\/-\/-\/-\/-\/-\/-\/-\/-

Observations 177

R2 0.299

Adjusted R2 0.287

Residual Std. Error 0.512 (df = 173)

F Statistic 24.636*** (df = 3; 173)

===============================================

Note: *p\textless0.1; **p\textless0.05; ***p\textless0.01

From Table 2, the estimated model is

log(\emph{salary})=.161 log(\emph{sales})+.098
log(\emph{mktval})+.00004\emph{profits}+4.687

\emph{profits} cannot be included in logarithmic form, because
\emph{profits} of 9 companies is negative.

The coefficient of \emph{profits} is estimated .00004. Hence, these firm
performance variables do

not explain most of the variation in CEO salaries.

(iii)Add the variable \emph{ceoten} to the model in part (ii). What is
the estimated percentage return for another year of CEO tenure, holding
other factors fixed?

\#model: add ceoten to C3.1

C3.2 \textless-
lm(log(salary)\textasciitilde log(sales)+log(mktval)+profits+ceoten,
data)

\#report the result

stargazer(C3.2, type = "text", title = "Table 3: sales\& mktval\&
profits\& ceoten")

Table 3: sales\& mktval\& profits\& ceoten

===============================================

Dependent variable:

-\/-\/-\/-\/-\/-\/-\/-\/-\/-\/-\/-\/-\/-\/-\/-\/-\/-\/-\/-\/-\/-\/-\/-\/-\/-\/-

log(salary)

-\/-\/-\/-\/-\/-\/-\/-\/-\/-\/-\/-\/-\/-\/-\/-\/-\/-\/-\/-\/-\/-\/-\/-\/-\/-\/-\/-\/-\/-\/-\/-\/-\/-\/-\/-\/-\/-\/-\/-\/-\/-\/-\/-\/-\/-\/-

log(sales) 0.162***

(0.039)

log(mktval) 0.102

(0.063)

profits 0.00003

(0.0002)

ceoten 0.012**

(0.005)

Constant 4.558***

(0.380)

-\/-\/-\/-\/-\/-\/-\/-\/-\/-\/-\/-\/-\/-\/-\/-\/-\/-\/-\/-\/-\/-\/-\/-\/-\/-\/-\/-\/-\/-\/-\/-\/-\/-\/-\/-\/-\/-\/-\/-\/-\/-\/-\/-\/-\/-\/-

Observations 177

R2 0.318

Adjusted R2 0.302

Residual Std. Error 0.506 (df = 172)

F Statistic 20.077*** (df = 4; 172)

===============================================

Note: *p\textless0.1; **p\textless0.05; ***p\textless0.01

From Table 3, the estimated model is

log(salary)=.162log (sales)+.102 log(mktval)+.00003 profits+.012
ceoten+4.558

The estimated percentage is 1.2\% for another year of CEO tenure,
holding other factors fixed.

(iv)Find the sample correlation coefficient between the variables
log(\emph{mktval}) and profits. Are these variables highly correlated?
What does this say about the OLS estimators?

\#View the scattergram

plot (log(data\$mktval),data\$profits, main="Correlation between
log(mkval) and profits", xlab="sales", ylab="profits")

\includegraphics[width=5.90556in,height=3.78542in]{media/image1.png}

\#calculate R

cor (log(data\$mktval),data\$profits)

The result on R shows correlation coefficient between sales and profits
is 0.7768976 and it says high correlation. High correlation between
variables makes variance large in OLS estimators.

\end{document}
